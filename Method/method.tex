%%%%%%%%%%%%%%%%%%%%%%%%%%%%%%%%%%%%%%%%%%%%%%%%%%%%%%%%%%%%%%%%%%%%%%%%%%%%%%%%
%2345678901234567890123456789012345678901234567890123456789012345678901234567890
%        1         2         3         4         5         6         7         8
% THESIS CHAPTER


\chapter[Control Architecture: Methods]{Control Architecture: Methods}
\label{chap:method}
\ifpdf
    \graphicspath{{Method/Figures/PNG/}{Method/Figures/PDF/}{Method/Figures/}}
\else
    \graphicspath{{Method/Figures/EPS/}{Method/Figures/}}
\fi

In this chapter, the theory explained in the previous chapter \ref{chap:control} is exploited to the specific scenario stated for this thesis.\\
A specific objective, the \textit{Force-Torque} objective (\ref{sec:forceTask}), is added to the TPIK list. This permits to help the insertion phase, dealing with forces and torques \textit{at kinematic level}.\\
Another section (\ref{sec:changeGoal}) describes another (additional) method to improve the performance of the mission. In brief, the goal frame inside the peg is, in general, not precise. This causes additional contacts between peg and inner hole. This collisions are used to modify the goal frame, reducing the error of the estimated goal pose.


\section{Force-Torque Objective}
\label{sec:forceTask}
Information from a force torque sensor can be exploited at kinematic level, inserting an additional control objective into the TPIK procedure. The aim of this objective is to zeroing the forces and torques acting on the peg. This is done generating properly joints and vehicle velocities to drive the peg in such a way the forces and torques decrease.\\
If we visualize the resultant of the forces caused by the collisions on the peg as a vector, moving \textit{linearly} the peg along this vector will cause the forces itself to decrease. The same idea can be use with torques, \textit{rotating}, along the resultant vector instead of moving linearly.\\

\noindent The \textit{feedback reference rate} for this objective will be:
\begin{equation}
	\label{eq:refForceTask}
	\boldsymbol{\dot{\bar{x}}}_{ft} \triangleq \begin{bmatrix}{\dot{\bar{x}}}_f \\ {\dot{\bar{x}}}_m \end{bmatrix} \triangleq 
	\begin{bmatrix} \gamma_f \\ \gamma_m \end{bmatrix} 
	\begin{bmatrix}
		0 - \| \boldsymbol{f} \| \\ 0 -\| \boldsymbol{m} \|
	\end{bmatrix} \qquad 0 < \gamma_f < 1, \quad 0 < \gamma_m < 1
\end{equation}
where $\| \boldsymbol{f} \|$ and $\| \boldsymbol{m} \|$ are the norms of the forces and torques vectors $\boldsymbol{f}$ and $\boldsymbol{m}$. Gains smaller that $1$ are necessary because big gains would mean too big velocities provided.\\
It can be noticed that, instead the full 3-dimensional vectors $\boldsymbol{f}$ and $\boldsymbol{m}$, the norms $f$ and $m$ are used. This is done to not overconstrain the system and to let more freedom to lower priority task. Even with norms, the control objective is obviously satisfied, because zeroing the norms brings each component of the vector to zero.\\

The \textit{feedback reference rate} of equation \eqref{eq:refForceTask} is intended to be like a velocity that the tool must follow. So, the Jacobian must be built considering this thing. For the \textit{task-induced} Jacobian (section \ref{sec:tpik}) of this new task, we have to take the linear and the angular part, and pre-multiplying them for the normal vector of $\boldsymbol{f}$ and $\boldsymbol{m}$ transposed:
\begin{equation}
	\boldsymbol{J}_{ft} \triangleq \begin{bmatrix}{\boldsymbol{J}}_f \\[1em] {\boldsymbol{J}}_m \end{bmatrix} \triangleq 
	\begin{bmatrix} \left( \dfrac{\boldsymbol{f}}{\| \boldsymbol{f} \|}\right)^T \enspace ^{lin}\boldsymbol{J}_t  \\[1.5em]
		\left( \dfrac{\boldsymbol{m}}{\| \boldsymbol{m} \|}\right) ^T \enspace ^{ang}\boldsymbol{J}_t  \end{bmatrix} 
\end{equation}
where $\boldsymbol{J}_f, \boldsymbol{J}_m \in \mathbb{R}^{1\times l}$; $J_t$ is the Jacobian which express how Cartesian velocity $\dot{\boldsymbol{x}}_t$ of the tool are affected by the system velocity vector  $\dot{\boldsymbol{y}}$; $lin, ang$ superscripts refer to \textit{linear} (top three rows) and \textit{angular} (bottom three rows) parts of $J_t$.\\

This objective can be considered as a \textit{pre-requisite} one (section \ref{sec:coClass}). So, it is put at higher priority than the \textit{reaching goal objective}. This will cause the robot to, first, try to nullified the forces and torques (if collisions happened), and only after (i.e. \textit{if possible}) to move the peg towards the goal.\\
Deactivating the task is necessary when the forces and/or torques are zero, to not generate system velocities for this task when they are not necessary. So a smooth activation function (section \ref{sec:activations}), $\boldsymbol{A} \in \mathbb{R}^{2 \times 2}$ is used:
\begin{equation}
	\begin{gathered}
		\boldsymbol{A}_{ft} \triangleq
		\begin{bmatrix}
			a_f & 0 \\
			0 & a_t \\
		\end{bmatrix} \\
		%	
		\vspace{10px}
		a_f(\| \boldsymbol{f}\|) \triangleq
		\begin{cases}
			0,& \| \boldsymbol{f}\| = 0\\
			s(\| \boldsymbol{f}\|), & 0 < \| \boldsymbol{f}\| \leq 0 + \Delta\\
			1, & \| \boldsymbol{f}\| > 0 + \Delta\\
		\end{cases} \\
		%		
		\vspace{10px}
		a_m(\| \boldsymbol{m}\|) \triangleq
		\begin{cases}
			0,& \| \boldsymbol{m}\| = 0\\
			s(\| \boldsymbol{m}\|), & 0 < \| \boldsymbol{m}\| \leq 0 + \Delta\\
			1, & \| \boldsymbol{m}\| > 0 + \Delta\\
		\end{cases}
	\end{gathered}
\end{equation}  
where $s(\cdot)$ is a smooth \textit{increasing} function from 0 to 1, and $\Delta$ a constant to create the smooth zone. 

\section{Objectives Prioritized List}
In this section, the objectives inserted into the TPIK procedure are listed and briefly explained.\\

The first task prioritized list, the one where the two robot acts independently to each other (section \ref{sec:coopScheme}), is:
\begin{itemize}
	\item \textbf{Joint Limits avoidance} (\textit{reactive, inequality, safety}): this objective keep joint away for their mechanical limits. It must be at high priority because it is a safe task, and also must be an inequality objective to not overconstrain the system when joints are away from their limits.
	
	\item \textbf{Horizontal Attitude} (\textit{reactive, inequality, safety}): to maintain the vehicle horizontal respect to the water surface. Most of the underwater vehicle are are passively stable in roll and pitch and these DOF are not controllable, so this objective is only needed for fully actuated vehicles (as stated in this case).
	
	\item \textbf{Force-Torque} (\textit{reactive, inequality, pre-requisite}): to nullified the forces and torques acting on the peg during the insertion. This objective is detailed in section \ref{sec:forceTask}.
	
	\item \textbf{Tool position control} (\textit{reactive, equality, mission}): this is the objective that define the mission. It is used to bring the tool towards the defined goal (i.e. inside the hole).
	
	\item \textbf{Preferred Arm Shape} (\textit{reactive, inequality, optimization}): this is a low priority objective that, \textit{if possible}, maintain the arm in a predefined shape. This shape permits the arm to have good dexterity but it is also useful to transport the peg in a natural way.
		
\end{itemize}

\noindent The categories (written in italic) are explained in section \ref{sec:controlObjectives}; further explanations on these and other objectives are available in \cite{IntroMaris2}, \cite{tesiWander}, \cite{IntroRecent}. Please note that in the code there is also an additional \textit{last task} which is used to cancel out any practical discontinuities during task activations [\cite{IntroMaris1}].\\

The TPIK procedure is run two more times, one for the vehicle-arm coordination (section \ref{sec:armVehScheme}), the other for the cooperation between the two robots (section \ref{sec:coopScheme}). Respectively, two \textit{non-reactive} objectives are put at the top of the hierarchy listed above, as explained in the cited sections.\\
%todo non serve più sta frase? The final output will be the velocity command $\dot{\boldsymbol{\bar{y}}}$.\\

Please note that some important objectives related to safe transportation (e.g. obstacle avoidance, minimum altitude from seafloor, minimum distance between robots), grasping (e.g. camera centring object) are not considered because they are not necessary in the particular experiment chosen, and also because they are explored in other works [\cite{IntroMaris2}; \cite{IntroRecent}].\\


\section{Changing the Goal Frame}
\label{sec:changeGoal}
In general, the frame where the tool is dragged by the control architecture (the \textit{goal} frame) is known with some errors. This is the case when, for example, we have some computer vision algorithm to estimate the hole pose. 
This error between ideal goal and estimated one, along linear and angular component, can cause the peg to collide more with the hole. If the peg clashes against the hole structure face (i.e. outside the proper hole) it is pushed back and a stuck may happen. In the literature, various methods (cited in \ref{sec:artPeg}) have been explored to deal with this problem.\\
This thesis focuses only the final part of the problem, i.e. when the peg is inside the hole, but bad alignment causes lot of collisions between peg and hole's internal walls. In this case, the peg usually does not stuck in a intermediate position, because the forces and torques acting on the peg \textit{naturally} drive it towards the goal (that is at a certain depth in the hole). In such a way, the peg continuously scrapes along the hole's walls, possibly damaging the pipe, the hole and also the robot which can suffer some stress. In practice there is a chattering problem: the peg continuously \textit{bounces} 
because the control wants to drag it towards the erroneous pose, while the hole's walls cause forces in opposite direction.\\
The method explained in this section try to solve this problem, modifying the goal accordingly to the forces acting on the peg. Considering the cartesian coordinate of the origin $^{w}\boldsymbol{g} \in \mathbb{R}^{3}$ of the goal frame, projected on the world frame, we modify it, providing $^{w}\boldsymbol{g'} \in \mathbb{R}^{3}$ as:
\begin{equation}
	\label{eq:changeGoal}
	\begin{gathered}
		%
	^{w}\boldsymbol{g'} = \,^{w}\boldsymbol{g} + \, ^{w}\boldsymbol{\tilde{f}} \\
	 ^{w}\boldsymbol{\tilde{f}} =  \,^{w}\boldsymbol{R}_t \,^{t}\boldsymbol{\tilde{f}} \\
	 ^{t}\boldsymbol{\tilde{f}} = k \, [ 0, \, f_y, \, f_z ] \qquad 0 < k < 1
%
	 \end{gathered}
\end{equation}
where $[ 0, \, f_y, \, f_z ]$ is the vector which represent the resultant of the forces acting on the peg, projected on the \textit{tool} frame, but with the component along x put to zero; $^{w}\boldsymbol{R}_t$ is the rotation matrix from world to tool; k a positive gain lesser than 1.\\
The component on $x$ axis of the force is neglected. This because the x axis of the tool frame $ \langle t \rangle $ is the one along the length of the peg. So, we do not want that this component modify the goal because can modify the wanted depth at which we want the peg to be inserted.\\
To proper utilize the force vector in the sum with the vector of the goal position, we must use a gain $k$. This gain is less than one change very slowly the goal position.\\
The formula \eqref{eq:changeGoal} is used every time a force (not null) is detected.

To understand better the method, we can take as an example one where the estimated goal is a bit on the left of the ideal one, but not so much to make the peg miss the hole. In such a case, the control architecture drive the peg on the left of the hole, causing a lot of collisions with the left side of the inner hole. So, the peg suffers forces with an important component in the right direction. Thus, this method shifts the goal to the right.\\
From this example should be noticed that this approach could cause the same problem in the opposite direction, if the goal is modified too much on the right. For this reason, setting a suitable $k$ is important.


